\documentclass[a4paper, 10pt]{article}

%-------------------------- Packete ------------------------------------
\usepackage[german]{babel}
\usepackage{graphicx}
\usepackage[utf8]{inputenc}
\usepackage{a4wide}
\usepackage{amsfonts}
\usepackage{amsmath}
\usepackage{amssymb}
\usepackage{fancyhdr}
\usepackage{color}
\usepackage{hyperref}
\usepackage{listings}
\usepackage{amsthm}
\usepackage{booktabs}
\usepackage[top=2cm, left=1.5cm, right=1.5cm, bottom=3cm]{geometry}
%------------------------- Einstellungen -------------------------------
\setlength{\parindent}{0pt}
\setlength{\parskip}{2ex plus 0.5ex minus 0.2ex}
\addtolength{\headheight}{1\baselineskip}
\setlength\parskip{\medskipamount}
\setlength\parindent{0pt}
\newcommand{\bs}{\ensuremath{\backslash}}

\lstset{
inputencoding=utf8,
extendedchars=true,
basicstyle=\footnotesize\ttfamily,
showstringspaces=false,
showspaces=false,
%  numbers=left,
%  numberstyle=\footnotesize,
numbersep=9pt,
tabsize=2,
breaklines=true,
showtabs=false,
captionpos=b,
literate={ö}{{\"o}}1 {ä}{{\"a}}1 {ü}{{\"u}}1 {°}{\dg}1 {»}{\frqq}1 {«}{\flqq}1 {ß}{\ss}1 {@}{\@}1 {Ä}{{\"A}}1
}
 
\theoremstyle{definition}
\newtheorem{definition}{Definition}[section]

%
%Edit the Course, assignment number and members of your group here
%
\newcommand{\courseName}{Berechenbarkeit und Komplexität}

\newcommand{\assignmentNr}{BuK Zusammenfassung WS18/19}


\newcommand{\studentB}{\textbf{Luis Rickert} -- 318153}
\newcommand{\kleene}{\Sigma^*}

%Left side of the Top header:
\fancyhead[L]{
	\textbf{\courseName}\\
	\assignmentNr{}
}

%Right side of the Top header:
\fancyhead[R]{
	\\	\studentB
}
\pagestyle{fancy}

\begin{document}
\small
\section{Einführung}
\subsection{Modellierung}
\begin{tabular}{l| l}
    $\epsilon$ & das leere Eingabe Wort\\
    \hline
    $\Sigma^0$ & Menge die nur das leere Wort enthält  \\
    \hline
    $\Sigma^k$ & Menge die alle Wörter der Länge k enthält\\
    \hline
    $\kleene$ & 'Kleenesche Abschluss' Menge die alle Wörter über $\Sigma$ enthalten\\
    \hline
    $\epsilon,0,1,00,01,11,000,001,010,011,100,101,\dots$&kanonische Aufzählung
\end{tabular}
\begin{itemize}
    \item Graphen über $\{0,1\}$ kodieren als Adjazenzmatrix
\end{itemize}
\subsection{Algorithmen}
\begin{itemize}
    \item verarbeiten Wörter schrittweise
    \item eindeutig festgelegt durch endlichen text
    \item zu jedem Algo. A mit Eingabe $w$ und Ausgabe $v$ wird eine funktion $f_A$ zugeordent mit $f_A(w)=v$
    \item terminiert $A$ nicht auf einder Eingabe, dann gilt für $f_A=$undefiniert sowie $f_A(w)=\perp$
\end{itemize}
\begin{definition}{Von Algorithmus berechnete Funktion }[Skript 1.6]\begin{itemize}
    \item Algorithms A, Eingabe $w$, Ausgabe $v$ wird funktion $f_A$ zugeordnet mit
    \item $f_A(w)=v$
    \item terminiert $A$ nicht auf $w$: \begin{itemize}
        \item $f_A(w)=$ undefiniert $\lor$
        \item $f_A(w)=\perp$
    \end{itemize}
\end{itemize}
\end{definition}
\begin{definition}{Berechenbare Funktionen }[Skript 1.7]
    \begin{itemize}
        \item Funktion $f$ mit Argumenten aus $M$\\
        $Def(f)=\{m\in M|f(m)\text{ ist definiert}\} \lor$\\
        $Def(f)=\{w\in\kleene|f(w)\neq\perp\}$\\
        'Definitionsbereich von $f$'\\
        $\rightarrow f$ bekommt seine Eingaben aus M, muss aber nicht auf allen definiert sein.
        \item $Bild(f)=\{n\in N|\exists m\in M:f(m)=n\}$\\
        'Bildbereich von $f$'
        \item $f:M\leadsto N$
    \end{itemize}
\end{definition}
\begin{definition}{Totale Funktion }[Skript 1.8]
\begin{itemize}
    \item Wie partielle Funktion nur das das gilt:
    \item $Def(f)=M,f:M\to N$
\end{itemize}
\end{definition}
\begin{definition}{Berechenbare Funktion }[Skript 1.10]\\
Eine Funktion $f:\kleene\leadsto\kleene$ heißt berechenbar gdw. es gibt einen Algorithmus $A$ der $f$ berechnet.
\end{definition}
\begin{definition}{Entscheibar }[Skript 1.12]\\
Sprache $L\subseteq \kleene$ heißt entscheidbar, wenn es einen Algorithmus gibt, der 'Ja' ausgibt wenn das Eingabewort $w$ L enthalten ist und 'Nein' wenn es nicht enthalten ist.
\end{definition}
\begin{definition}{Aufzählungsalgorithmen }[Skipt 1.13]\\
Aufzählungsalogrithmus $A$ startet auf $\epsilon$ und braucht nicht zu terminieren. Er gibt in irgenteiner Reihenfolge Wörter aus. Diese fasst man in der Menge $L_A$ zuammen.\\
Einge Menge von Wörtern oder Sprache L heißt aufzählbar wenn es  einen Aufzähler $A$ gibt mit $L=L_A$.\\ \\
 
$\Rightarrow$ \underline{aufzählbarkeit $\neq$ entscheidbarkeit ! }

\end{definition}\newpage
\subsection{Turingmaschinen}
$\rightarrow$ abstraktes Automatenmodell, entwickelt von Alan Turing, um die Durchführung beliebiger Algorithmen zur Symbolmanipulation präzise zu fassen.
\begin{definition}{Turingmaschine (TM)}[Skript 1.3.1]\\
\begin{itemize}
\item $Q $  endliche Zustandsmenge
\item $\Sigma\supseteq\{0,1\}$ endliche Eingabealphabe
\item $\Gamma\supset\Sigma$,  endliche Bandalphabet
\item $B\in\Gamma/\Sigma$, Leerzeichen (Blank)
\item $q_0\in Q$, Anfangszustand
\item $\bar{q}\in Q$, Stopp/Endzustand 
\item Zustandsübergangsfunktion:\begin{align*}
	\delta:(Q/\{\bar{q}\}\times\Gamma\to Q\times\Gamma\times\{R,L,N\}
\end{align*}
\end{itemize}
\begin{enumerate}
	
\end{enumerate}
\end{definition}
\subsubsection{Konfiguration}

\paragraph{Konfiguration}
\begin{itemize}
\item Ist ein String $\alpha q\beta$ ,wobei $q\in Q \land \alpha,\beta \in\Gamma^*\to$ man ist im Zustand $q$ und auf dem Band steht $\alpha\beta$ eingerahmt von Blanks.
\end{itemize}
\paragraph{direkte Nachfolgekonfiguration}
\begin{itemize}
\item $\alpha'q\beta'$ heißt \underline{direkte} Nachfolgekonfig. wenn aus $\alpha q\beta$ in  einem Rechenschritt $\alpha'q\beta'$ wird. $\to\alpha q'\beta \vdash \alpha'q\beta'$
\end{itemize}
\paragraph{Nachfolgekonfiguration}
\begin{itemize}
\item $\alpha'' q''\beta''$ heißt Nachfolgekonfiguration von $\alpha q\beta$ wenn $\alpha'' q''\beta''$ in endlich vielen Rechenschritten erreicht werden kann $\to\alpha q\beta \vdash \alpha''q''\beta''$
\end{itemize}
\paragraph{TM-Berechenbare Funktionen}
- TM die für jede Eingabe terminiert berechnet eine totale Funktion.
- Das die Ausgabe beginnt unter dem Lese/Schreibkopf und wird nach  rechts durch den ersten Buchstaben begrenzt der nicht im Bandalphabet ist.
Da die meisten TM nicht immer terminieren, berechnen im allegmeinen eine Funktion $F:M\leadsto N$.
\paragraph{TM-Berechenbarkeit, Rekursivität}
Eine funktion heißt TM-berechenbar oder rekursiv, wenn es eine TM gibt, die auf jeder Eingabe $w$ terminiert mit der Ausgabe $f(w)$ .
\paragraph{partiel rekursiv} Eine Funktion $f$ heißt paritell rekursiv, wenn es eine TM M gibt die, genau auf den Wörtern $w\in Def(w)$ terminiert mit der Ausgabe $f(w)$.
\begin{definition}{TM-Entscheibarkeit}[Skript 1.17]\\
Eine Sprache $L$ heißt TM-Entscheidbar oder rekursiv, wenn es eine TM $M$ gibt, die auf allen eingaben $w\in\kleene$ stoppt und die Eingabe $w$ genau dann akzeptiert, wenn $w\in L$ und die Eingabe verwirft wenn $w\not\in L$.
\end{definition}
\begin{definition}{Aufzählungs-Turingmaschine}[Skript 1.18]
\begin{itemize}
\item TM der Form $(Q,\Sigma,\Gamma,B,q_0,bar{q},q_{out},\delta)$
\item $q_{out}$ deints als Ausgabezustand für die einzelnen Wörter, das an Kopfposition beginnt und vor dem ersten Symbol aus $\Gamma /\Sigma$ endet.
\end{itemize}
Eine Sprache $L$ heißt rekursiv aufzählbar, wenn es eine Aufzähler gibt, der die Spache $L$ enthält
\end{definition}

\subsection{Zusammenfassung Kaptiel 1}
\begin{itemize}
\item eine Funktion ist berechenbar wenn es eine TM gibt, die sie berechnet
\item eine Funktion ist total, wenn sie für alle Möglichen Eingaben definiert ist $f:M\to N$
\item eine Funktion ist partiell, wenn sie für eine Teilmenge der möglichen Eingaben definiert ist $f:M\leadsto N$
\item eine Sprache ist rekursiv, wenn es eine TM gibt die sie entscheidet\begin{itemize}
\item $1\forall w \in L$
\item $0\forall w\not\in L$
\end{itemize}
\item eine Sprache ist paritell rekursiv, wenn es eine TM gibt die auf alle ihren Wörtern hält und diese akzeptiert\begin{itemize}
\item $f_M(w)=v\forall w\in L$
\item $f_M(w)=\perp\forall w\not\in L$ ($\perp\widehat{=}$terminiert nicht bzw nicht definiert)
\end{itemize}
\item eine Sprache ist aufzählbar(=rekursiv aufzählbar), wenn es einen Aufzähler für sie gibt
\end{itemize}
\end{document}